\documentclass[itdr/core]{subfiles}

\begin{document}

\chapter{Treasure and Magic}
\label{ch:treasure_and_magic}
\index{Treasure}
\index{Money}
\index{Magic}

\paragraph{Riches}
Different types of treasure, from gems to artwork to functional items, have a certain value. Traders often want to haggle this price or, in the case of items worth thousands of Shillings, they may not be able to afford it at all.

\paragraph{Coins}
All coins are valued against the \textbf{Silver-Standard Shilling} (s). One Shilling gets you a decent bed, meal and drink for the night and is the amount a typical labourer earns in a week.

There is a huge variety of coins that are valued against the Shilling, with the following two being especially common.\tight

\subparagraph{Copper Pennies} (p) are worth a tenth of a Shilling. One Penny buys you a cheap drink in a bad tavern or a passage on a ferry.

\subparagraph{Gold Guilders} (g) are worth one hundred Shillings. One Guilder gets you a good horse, a full set of armour, or a valuable piece of jewellery.

\vfill

\paragraph{Creating New Magic Spells}
\index{Spells}
Use \textbf{\fullref{ch:magic}} as a reference of power levels and possible effects when creating new Spells.

\index{Spells!damage}
Rough Damage estimate:
\begin{itemize}
	\item \textbf{Cantrips:} d4
	\item \textbf{\nth{1} Circle:} d4 to d6
	\item \textbf{\nth{2} Circle:} d6 to d8
	\item \textbf{\nth{3} Circle:} d8 to d10
	\item \textbf{\nth{4} Circle:} d10 to d12
	\item \textbf{\nth{5} Circle:} d12
\end{itemize}

Continuous and area-of-effect Spells usually deal less Damage than instant ones of the same Circle.

\index{Damage!elemental}
\index{Elemental Damage|see {Damage, elemental}}
Some Spells might deal Elemental Damage. The~most common are Cold, Electricity, and Fire.

Appropriate saves against certain effects:
\begin{itemize}
	\item \textbf{STR:} physical obstacles, touch Spells, metamorphosis and other bodily influences
	\item \textbf{DEX:} evasion, balance, extinguishing the flames
	\item \textbf{WIL:} mind control: charm, fear, illusions, etc.\note
\end{itemize}

\notetext{Undead-affecting magic does not count as mind control for the purposes of resistances and immunities.}

\vfill

\paragraph{Breaking the Rules}
Not all magic functions as that of Mystics. Magic can do anything and is not subject to limitations.

\vfill
\break

\paragraph{Magic Weapons and Armour}
\index{Weapons}
\index{Armour}
\index{Runic}
Weapons created with magical power often have Runic symbols engraved on them, telling their name, history, and purpose. As well as having a \textbf{Raised Damage die} (up to d10) and \textbf{ignoring supernatural resistances}, magical weapons will have an \textbf{extra feature}, such as bursting into flames when it draws blood or guiding the wielder towards gold. This will never be a matter of simply \mbox{doing} more Damage, though some weapons may cause \mbox{\textbf{additional effects}} when they cause Critical Damage, such as turning the victim to stone.

Similarly, magic armour and shields will have an \textbf{extra feature} or offer \textbf{greater protection} against a specific source of Damage.

\vfill

\paragraph{Magic Items}
\index{Magic!items}
Other magic items could include rings, cloaks, gloves, and pendants. These may grant a \textbf{continual effect} on the wearer or require \textbf{activation}. The effect will usually not be exactly the same as a Spell but may be similar.

\vfill

\index{Consumables|see {Magic, items}}
\subparagraph{Consumable Magic Items} such as potions will grant a one-off benefit to the consumer.

\vfill

\index{Rings|see {Magic, items}}
\subparagraph{Magic Rings} are limited to one ring per hand.

\vfill

\index{Wands|see {Magic, items}}
\index{Rods|see {Magic, items}}
\subparagraph{Wands and Rods} have a limited and unknown number of charges. After the first use, roll a d4 and write it down. Every time you use the item, roll a d6. If you roll over this number, decrease it by one. On zero, the item is drained and becomes useless.

\vfill

\paragraph{Drawbacks and Curses}
Most powerful magic items usually have some kind of a drawback or a detriment to their user, either \mbox{permanent} or occurring each time the item is used. Such properties cannot be revealed through \textit{Identify} Spell but only through experimentation and usage.

\vfill
\dimage{treasures}{112pt}
\vfill
\break


\section{Example Magic Items}
\index{Magic!items}

\paragraph{Amulet of Health Protection}
When found, this ruby amulet has Power of 2d6+6.

Any Damage to STR~Score is subtracted from the amulet's Power instead, then roll a d20: if the roll exceeds amulet's power, it cannot be used again today. Once Power reaches 0, the amulet shatters to pieces.

\vfill
\paragraph{Cloak of Descent}
This leather cloak slows down the falling speed and even allows its user to stir and glide a small distance.

\vfill
\paragraph{Diadem of Empathy}
A thin glass diadem allows its wearer to sense the true feelings and emotions of others.

\vfill
\paragraph{Flying Broom}
When mounted, this broom can carry up to two \mbox{humans}. It can be used as a Mystic's Focus as well.

\vfill
\paragraph{Flying Carpet}
This peculiarly patterned carpet is feather-light and can carry up to 8 humans into the air (but only half as fast as a flying broom).

\vfill
\paragraph{Helm of Breathing}
If needed, this crystal helm provides its wearer with a clean air supply for up to one hour.

\vfill
\paragraph{Mask of Disguise}
This sleek silver mask allows its user to assume \mbox{facial} appearance of others once per day.

\vfill
\paragraph{Ring of Regeneration}
This willow ring restores 1 STR~Loss per day.

\vfill
\paragraph{Spider Silk Gloves}
Made of enchanted spider silk, these elegant gloves allow the wearer to climb any surface. The same adhesive property might impose Advantage or Disadvantage on appropriate Saves as well.

\vfill
\paragraph{Turnskin}
This animal skin turns its wearer into a corresponding creature. Each time the character wears it, roll a d100. On 1, the turnskin cannot be taken off until the curse is removed. The chance increases by 1\% for each subsequent use by the same character.

\vfill
\break

\subsection*{Armour and Weapons}

\paragraph{Cobra Staff}
This carved staff (d8, two-handed) ends with a stylized cobra head. Along with the Damage, it deals d4 DEX~Score~Loss (affected by Armour) as well.

\paragraph{Ironwood Armour}
Any Electricity Damage cannot ignore this full \mbox{armour} made of unnaturally strong dark wood.

\paragraph{Lucky Boomerang}
This exotic ivory boomerang always finds its target thus negating Impairments from cover and such.

\paragraph{Mirror Shield}
This mirror-polished steel shield has a chance to block an incoming Spell based on its circle: \mbox{0--1: 3-in-6, 2--3: 2-in-6, 4--5: 1-in-6}. A blocked Spell has a 2-in-6 chance of reflecting back to the caster.

\vfill

\subsection*{Consumables}

\index{Poison}
\paragraph{Deadly Poison}
This dark oily liquid deals d6 STR~Loss prompting a Critical Damage Save if consumed. On a failed Save, the consumer dies. When applied to a suitable weapon or a set of projectiles, Critical Damage Saves from it are made at Disadvantage until the next Rest.

\paragraph{Four-Leaf Clover}
Reroll one failed Save, then the clover withers away.

\paragraph{Health Potion}
The vial of sparkling red liquid restores d6 STR~Loss.

\paragraph{Needle of Negation}
When this thin silver needle is broken, it disrupts ongoing Spell effects in a small area for a minute.

\vfill

\subsection*{Wands and Rods}

\paragraph{Rod of Reveal}
This obsidian rod reveals illusions, invisible entities, secret doors, traps, etc. in the direction it is pointing.

\paragraph{Wand of Shock}
This amber wand deals d6 Electricity Damage \mbox{ignoring} Armour.

\vfill

\begin{dbox}
	See \textbf{\safenameref{subsec:random_magic_items}[Random Magic Items]} in \textbf{\customref{ch:appendix_a}{Appendix A}} for \mbox{additional inspiration}.
\end{dbox}

\vfill

\end{document}
