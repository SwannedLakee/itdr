\documentclass[itdr/core]{subfiles}

\begin{document}

\cleartoleftpage

\chapter{Monsters}
\label{ch:monsters}
\index{Monsters}

Monsters are, by their very nature, different to people and animals. Thus they often have special abilities outside of their Ability Scores. A dungeon should contain mostly unique monsters but some examples are given in the \textbf{\customref{ch:appendix_b}{Appendix B: Bestiary}}.

\vfill
\index{Hit Points}
\paragraph{Hit Points}
Most creatures have between 1d6 and 5d6 HP. Remember that Hit Points are not purely the ability to absorb physical damage but also the monster's cunning and skill in avoiding harm.

\vfill
\paragraph{Killing Monsters}
Monsters are treated exactly the same as characters other than noted exceptions.

\vfill
\index{Magic!monsters}
\paragraph{Magic}
While some monsters may use Spells in the same way as Mystics, some are able to use Spells without a Tome or Focus. Monsters do not need to follow the rules.

\vfill
\index{Armour}
\paragraph{Armour}
Use character armour as a guide for how to represent monsters with tough hides or those large enough to be able to shrug off most weapons.

\vfill
\index{Damage!monsters}
\paragraph{Damage}
Most monsters cause d6 Damage if nothing is mentioned. Some have a bigger Damage die or even bonus Damage dice.

\vfill
\index{Ability Score Loss}
\index{Death}
\paragraph{Ability Score Loss and Death Attacks}
Particularly deadly creatures may reduce the target's Ability Score if they cannot make a Save, often resulting in a horrible fate if the score is reduced to zero.

\vfill
\index{Ability Scores}
\paragraph{A Note on Ability Scores}
When assigning Ability Scores, 20 should generally be considered the maximum. A huge monster may look like it should have a STR of 30 or more, but consider that large creatures may not fight all that well. They should instead have their size represented by dealing more Damage and having a higher Armour score.

\vfill
\break

\section{Monster Conversion}
\index{Monster Conversion}

\subsection*{\nth{5} Edition}

\subparagraph{HP:} 1hp per HD. Maximum of 30. If no HD is specified, HD~=~HP/(5+CON~Modifier) (round down).
\subparagraph{Armour:} Increase by 1 for noted armour, extreme resilience, and each size category above Medium.
\subparagraph{Ability Scores:} Directly transferable, use CHA for WIL. Maximum of 20.
\subparagraph{Attacks:} Start at d6. Raise by one die for each size category above Medium and once more if they wield a heavy weapon. No multi-attacks.
\subparagraph{Vulnerability / Resistance:} Replace with Enhance / Impair respectively.

\vfill

\subparagraph{Other Editions:} Same as \nth{5} edition except:
\subsection*{\nth{4} Edition}
\subparagraph{HP:} 1hp per Level. $\times$3 for Solo creatures, +1hp for Small or bigger creatures.
\subparagraph{Ability Scores:} Same as 5e, except:
\begin{itemize}
	\item --4~STR for Humanoids and Monstrosities
	\item --2~STR for Undead
	\item --4~DEX for Large or bigger creatures
	\item --2~DEX for Medium or smaller Humanoids and Undead
	\item --2~WIL for Monstrosities
\end{itemize}

\vfill
\subsection*{\nth{3} and 3.\nth{5} Editions}
\subparagraph{HP:} 1hp per HD. +1hp for Small or Medium creatures and +2hp for Large or bigger creatures, except Oozes.
\subparagraph{Ability Scores:} If STR is not specified --- below 10.

\vfill
\subsection*{Original, Basic, and Advanced Editions}
\subparagraph{HP:} 1hp per HD. +1hp for Small and Medium creatures and Large or bigger Oozes; +2hp for Large or bigger creatures.

\subparagraph{Morale:} keep using 2d6 (Original and Basic), 2d10~(Advanced), or convert it to d20 (WIL):

\begin{dtable}[CCC|CCC]
	\textbf{2d6} & \textbf{2d10} & \textbf{d20} & \textbf{2d6} & \textbf{2d10} & \textbf{d20} \\
	2 & 2--3 & 1	& 7 & 11--12 & 11--13 \\
	3 & 4--5 & 2	& 8 & 13--14 & 14--16 \\
	4 & 6--7 & 3--4	& 9 & 15	 & 17 \\
	5 & 8	 & 5--6	& 10& 16--17 & 18 \\
	6 & 9--10& 7--9 & 11& 18--19 & 19 \\
\end{dtable}
\vfill
\break

\section{Ideas for Monster Creation}

\paragraph{Appearance and Behaviour}
Change the visual appearance and behaviour of the existing monster. Changing the size or combining a couple of monsters into one is also a possibility.

\vfill
\paragraph{Characters' Features}
Apply Features from \textbf{\fullref{ch:characters}} to non-player-characters and monsters, especially ``bosses''.

\vfill
\paragraph{Effect on Critical Damage}
\index{Damage!critical}
On a failed Critical Damage Save, a monster's target suffers some additional detrimental effect: illness, poison, ability score loss, or even death. Decide if the target could Save against this.

\vfill
\paragraph{Pairing}
One type of monsters enhances other type's attacks, provides protection or some other advantage.

\vfill
\paragraph{Power-ups}
A monster receives a power-up, a new attack, or changes tactics when it runs out of HP, saves against Critical Damage for the first time, takes Damage from a specific source, etc.

\vfill
\paragraph{Special Abilities and Attacks}
Instead of its default attack, a monster can use a special one, be it a Spell-like ability or some other unusual effect. Some of these abilities might be ``passive'' (always enabled).

\vfill
\paragraph{Tactics and Weapons}
Monsters might use unexpected combat tactics, especially when they fight in groups. If a monster is armed, change its weapon to something unusual or switch the weapon's melee/ranged type.\tight

\vfill
\paragraph{Vulnerabilities, Resistances, and Immunities}
\index{Attacks!enhanced}
\index{Attacks!impaired}
Specific attacks against the monster are Enhanced, Impaired, or do not work at all.

\vfill
\begin{dbox}
	See \textbf{\customref{ch:appendix_b}{Appendix B: Bestiary}} for example monsters and additional inspiration.
\end{dbox}

\vfill
\break

\section{Example Monster Abilities}

\paragraph{Absorption}
When a monster takes Damage from a certain source (usually, an elemental one), it restores the monster's HP (or even STR) for the value of this Damage instead.\tight

\vfill
\paragraph{Charge}
A monster rapidly closes the distance to its target. The target must succeed on a \save{DEX} or suffer increased Damage and/or other effects.\tight

\vfill
\paragraph{Extra Limbs}
A monster has multiple Damage dice (still taking the highest one for a single target). Some monsters can even attack multiple opponents, dividing Damage dice between these attacks.

\vfill
\paragraph{Grapple}
If a target fails a \save{DEX}, it is Restrained until a~successful \save{STR or DEX} on the following turns. Monsters cannot attack with limbs they are currently using for grappling, but strong ones might damage the grappled target instead.

\vfill
\paragraph{Indomitable}
Once per Rest, when taking Critical Damage, a monster continues to fight as if it succeeded on this Save. Some artificial or undead monsters might ignore Critical Damage effects altogether.

\vfill
\paragraph{Swallow}
The target must succeed on a \save{DEX} or be swallowed whole, suffering Ability Score Loss (STR, DEX, or both) each following turn. If the monster suffers Critical Damage, it must pass an additional \save{STR} or regurgitate all swallowed creatures.

\vfill
\paragraph{Volatile}
When a monster suffers Critical Damage, it explodes, dealing Blast Damage to everyone nearby.

\vfill
\paragraph{Weakness}
When a monster takes Damage from a source of its weakness (even if this Damage is not the highest one this turn), the monster loses some of its powers, becomes Stunned, etc. Usually, such an effect lasts for the monster's next turn.\tight

\vfill

\end{document}
