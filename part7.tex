\documentclass[itdr]{subfiles}

\begin{document}

\chapter{Hazards and Obstacles}
\label{ch:hazards_and_obstacles}
\index{Hazards}
\index{Traps}
\index{Obstacles}
\index{Exploration}

\paragraph{Spotting Hazards}
Generally, the characters always notice the presence of a trap or other hazard unless they are running, distracted, or visually impaired. After this, the characters may be harmed through further inaction or lack of caution. The players should consider creative ways of avoiding or disarming a hazard. Risky methods may call for a Save or Luck Roll.

%\vfill
\paragraph{Locked Doors}
\index{Doors}
Typically, a locked door can be opened with a lockpick, given some time. No Save is required unless there is a risk of triggering a trap, alerting foes, or running out of time.

Using lockpicks and other equipment quickly or under pressure usually require a \save{DEX} and may include having to light a torch while under attack or tying a rope before a friend plummets to their death.

Breaking down a door can also be done without a Save unless there are risks or pressure, which may require a \save{STR}. However, breaking down a door always causes lots of noise and can take a long time.

%\vfill
\paragraph{Random Encounters}
\index{Encounters}
\index{Random Encounters|see {Encounters}}
Anything mobile in a dungeon is unlikely to remain in one place all the time. As such, the Referee should consider having a chance of the group encountering someone or something. Making loud noises increases or decreases the chance of this happening, depending on the nature of the encounter.

When characters explore, rest, cast unprepared Spells, or hesitate in a dangerous place, roll a \textbf{d6}.

\begin{dtable}[cL]
	\textbf{Roll} & \textbf{Outcome} \\
	1	& Roll for a Random Encounter.\\
	2	& Roll for a Random Encounter. Give a sign that it is nearby or has passed through.\\
	3+	& Clear.\\
\end{dtable}

Delaying for long enough to have a meal or sleep provokes a \textbf{d4} roll instead.
Larger dice (\textbf{d8} to \textbf{d12}) could be used for less dangerous environments.

When travelling through the wilderness, check for a Random Encounter at least once daily and nightly.

%\vfill
\paragraph{Random Reaction}
\index{Reaction}

If encountered creatures' reaction is \textbf{uncertain}, roll a die, based on their general disposition, from d4 (peaceful) to d12 (aggressive). If parleying, passing a \save{WIL} will grant a Lowered reroll.

\begin{dtable}[cL|cL]
	\textbf{Roll} & \textbf{Reaction} & \textbf{Roll} & \textbf{Reaction} \\
	1 & Friendly	& 3	& Cautious\\
	2 & Indifferent	& 4+& Hostile\\
\end{dtable}

\vfill

\section{Example Random Encounters}
\index{Encounters}

\header{Dungeon Encounters}
\begin{dtable}[cL]
	\textbf{2d4} & \textbf{Encounter} \\
	2	&	gelatinous cube	\\
	3	&	d4 rust monsters	\\
	4	&	d8 skeletons	\\
	5	&	2d6 goblins	\\
	6	&	d6 orcs	\\
	7	&	filth eater	\\
	8	&	hook horror	\\
\end{dtable}

\vfill

\header{Wilderness Encounters}
\begin{dtable}[cL]
	\textbf{d4+d6} & \textbf{Encounter} \\
	2	&	ogre	\\
	3	&	runaway horse	\\
	4	&	2d6 goblins, a 2-in-6 chance of ambush	\\
	5	&	d6 huntsmen	\\
	6	&	pack of 3d4 wolves	\\
	7	&	wild boar	\\
	8	&	pack of 3d6 wolves	\\
	9	&	d4 deer	\\
	10	&	bear	\\
\end{dtable}

\vfill

Random encounter tables can be used in a friendly environments as well.

\header{Urban Encounters}
\begin{dtable}[cL]
	\textbf{2d8} & \textbf{Encounter} \\
	2	&	street brawl; a 2-in-6 chance that watchmen are already present	\\
	3	&	brash urchin tries to steal a purse or some random item from a character	\\
	4	&	group of servants carrying a palanquin	\\
	5	&	travelling merchant selling exotic goods	\\
	6	&	drunkard looking for trouble	\\
	7	&	loud advertiser for a nearby establishment	\\
	8	&	crippled beggar at the street corner	\\
	9	&	street food merchant	\\
	10	&	broken cart blocking the road	\\
	11	&	city watch patrol of 2d4 watchmen	\\
	12	&	band of street performers	\\
	13	&	priest collecting charity for a local temple	\\
	14	&	watchmen escorting a caught thief	\\
	15	&	local holiday parade	\\
	16	&	ambush (2d4 criminals) in the dark alley	\\
\end{dtable}

%\vfill
\break

\section{Example Traps}
\index{Traps}

\paragraph{Balancing Ledge}
Must be crossed to reach whatever lies on the other side. Can be done quite safely without pressure, but when having to run or under attack, make a \save{DEX} or fall to the lower level, requiring a rope to climb back up.

The lower level contains crocodiles (STR~13, DEX~5, WIL~5, 9hp, Armour~1, d8~Bite).

\vfill
\paragraph{Cage Pit}
A trapdoor is visible unless the character is distracted, sprinting, or the vision is impaired. Triggered by stepping onto the trapdoor.

Triggering the trap causes d8 Damage, a metal cage traps the victim until released with a key, and an alarm mobilizes someone unpleasant.

\vfill
\paragraph{Grasping Vines}
Triggered on nearing strange-looking vines. Take d6 Damage each turn until you break free. \save{STR} to break free each turn, otherwise you are immobile. Highly flammable.

\vfill
\paragraph{Stupefying Dart Trap}
A dart pipe is visible at the base of the chest. Triggered by opening the chest without taking appropriate precautions. Broken darts litter the floor of this room.
d8 Damage. d8 DEX Loss on Critical Damage.

\vfill
\paragraph{Swinging Blade Trap}
Eternally swinging over a corridor in a sequence. Can be blocked only by a very strong metal pole or other suitable objects.

\save{DEX} to pass through without harm, otherwise taking d10 Damage while passing through.

\vfill
\paragraph{Traitor's Circle}
Triggered by entering the circle marked with a symbol depicting a dagger being thrust into a heart.

\save{WIL} or immediately attack your closest ally, continuing until knocked Unconscious. If you pass this Save, you are thrown out of the circle and take d6 Damage.

\vfill
\break

\section{Example Obstacles}
\index{Obstacles}

\paragraph{Control Room}
A room full of levers and buttons that switch corridors, gates, and hidden devices throughout the dungeon. No markings or instructions present.

\vfill
\paragraph{Crystal Floor}
A floor is made of a crystal material smoother than ice. Movement is highly difficult, and a risk of falling and sliding down a slope is everpresent.

\vfill
\paragraph{Distorted Dimensions}
The dungeon does not follow the common laws of geometry as it exists in a different set of dimensions.

\vfill
\paragraph{Flying Fortress}
An ancient structure that floats at an unreachable height, following a daily route, sometimes passing pretty close to the local mountain range.

\vfill
\paragraph{Gravitational Anomaly}
A zone of altered gravity (direction or strength).

\vfill
\paragraph{Magic Negation Sphere}
A mysterious device on top of the colossal stone spire sucks out magic energy thus disabling Spells and magic objects the closer you get to it, starting with \nth{5} Circle Spells and leaving Mystics with just their Cantrips in the nearest proximity to it. Magic items have a reduced chance of successful operation as well (from 5-in-6 to 1-in-6 chance accordingly).

\vfill
\paragraph{Mind Barrier}
A wall of force that exclusively blocks conscious sentient beings.

\vfill
\paragraph{Remote Activation}
A portcullis that opens by turning the wheel in the nearby chamber.

\vfill
\paragraph{Underwater Passage}
A flooded room with a tunnel at the bottom.

\vfill
\paragraph{Unfinished Tunnel}
There is an undiscovered cave behind just a couple feet of rock. Sounds or some other signs might suggest its presence.

\vfill

\end{document}
